\documentclass{beamer}
\usetheme{Madrid}

%Packages BEGIN
\usepackage{amsmath}
\usepackage{physics}
\usepackage{hyperref} 
\usepackage{mathtools} 

%Packages END

% Chinese settings start
\usepackage{xeCJK} % use this package to set Chinese and English font separately
\setCJKmainfont{Noto Serif CJK TC} % Serif font in Ubuntu. Choose the Chinese font available in your device
\setCJKmonofont{Noto Serif CJK TC} % Serif font in Ubuntu. Choose the Chinese font available in your device
\setCJKsansfont{Noto Serif CJK TC} % Serif font in Ubuntu. Choose the Chinese font available in your device
\XeTeXlinebreaklocale "zh" % enabling auto linebreaks
\XeTeXlinebreakskip = 0pt plus 1pt % enabling auto linebreaks 
% Chinese settings end

\usefonttheme[onlymath]{serif} 
\setbeamertemplate{bibliography item}{\insertbiblabel} 

\title[Factorizing 15]{Factorizing 15 Using Shor's Algorithm}
\author{Hao-Chien Wang}
\institute[NTUPhys]{Department of Physics, National Taiwan University}

\begin{document}

\begin{frame}
	\titlepage
\end{frame}

\begin{frame}{Outline}
	\tableofcontents
\end{frame}

\section{Review}%
\label{sec:review}

\begin{frame}{Review}
	
\end{frame}

\section{Method}%
\label{sec:method}

\begin{frame}{Method}

\end{frame}

\subsection{Easy case}%
\label{sub:easy_case}

\begin{frame}{Easy case}
	
\end{frame}

\begin{frame}{Easy case: Circuit}
	
\end{frame}

\subsection{Difficult case}%
\label{sub:difficult_case}

\begin{frame}{Difficult case}
	
\end{frame}

\begin{frame}{Difficult case: Circuit}
	
\end{frame}

\section{Results}%
\label{sec:results}

\begin{frame}{Result: Easy case}
	
\end{frame}

\begin{frame}{Result: Difficult case}
	
\end{frame}

\section{Discussion}%
\label{sec:discussion}

\begin{frame}{Discussion}
	
\end{frame}

\end{document}
